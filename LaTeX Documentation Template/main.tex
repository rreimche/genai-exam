\documentclass[]{reflection}
\usepackage{eso-pic} 
\usepackage{lipsum}
\usepackage{svg}
\usepackage{titleps}
\usepackage{hyperref}
\usepackage[nointegrals]{wasysym}

% Specify that the source file has UTF8 encoding
\usepackage[utf8]{inputenc}
% Set up the document font; font encoding (here T1) has to fit the used font.
\usepackage[T1]{fontenc}
\usepackage{lmodern}
\usepackage{ragged2e}

% Load language spec
\usepackage[english]{babel} 
% German article --> ngerman (n for »neue deutsche Rechtschreibung«)
% British English --> english

% Ffor bibliography and \cite
\usepackage{cite}

% AMS extensions for math typesetting
\usepackage[intlimits]{mathtools}
\usepackage{amssymb}
% ... there are many more ...


% Load \todo command for notes
\usepackage{todonotes}
% Sebastian's favorite command for large inline todonotes
% Caveat: does not work well with \listoftodos
\newcommand\todoin[2][]{\todo[inline, caption={2do}, #1]{
		\begin{minipage}{\linewidth-1em}\noindent\relax#2\end{minipage}}}

\newcommand{\funfact}[2]{\noindent\fbox{%
    \parbox{\textwidth}{%
        \textbf{#1:} #2
    }%
}}
% Load \includegraphics command for including pictures (pdf or png highly recommended)
\usepackage{graphicx}

% Typeset source/pseudo code
\usepackage{listings}

% Load TikZ library for creating graphics
% Using the PGF/TikZ manual and/or tex.stackexchange.com is highly adviced.
\usepackage{tikz}
% Load tikz libraries needed below (see the manual for a full list)
\usetikzlibrary{automata,positioning}

% Load \url command for easier hyperlinks without special link text
\usepackage{url}

% Load support for links in pdfs
\usepackage{hyperref}

% Defines default styling for code listings
\definecolor{gray_ulisses}{gray}{0.55}
\definecolor{green_ulises}{rgb}{0.2,0.75,0}
\lstset{%
  columns=flexible,
  keepspaces=true,
  tabsize=3,
  basicstyle={\fontfamily{tx}\ttfamily\small},
  stringstyle=\color{green_ulises},
  commentstyle=\color{gray_ulisses},
  identifierstyle=\slshape{},
  keywordstyle=\bfseries,
  numberstyle=\small\color{gray_ulisses},
  numberblanklines=false,
  inputencoding={utf8},
  belowskip=-1mm,
  escapeinside={//*}{\^^M} % Allow to set labels and the like in comments
}

% Defines a custom environment for indented shell commands
\newenvironment{displayshellcommand}{%
	\begin{quote}%
	\ttfamily%
}{%
	\end{quote}%
}

%%%%%%%%%%%%%%%%%%%%%%%%%%%%%%%%%%%%%%%%%%%%%%%%%%%%%%%%%%%%%%%%%%%%%%%%%%%%%%%

\title{TaskTitle}
\event{Data Science and its Applications WiSe 2024-25}
\author{Firstname Lastname
  \institute{RPTU Kaiserslautern, Department of Computer Science}}

%%%%%%%%%%%%%%%%%%%%%%%%%%%%%%%%%%%%%%%%%%%%%%%%%%%%%%%%%%%%%%%%%%%%%%%%%%%%%%%
\begin{document}
%%%%%%%%%%%%%%%%%%%%%%%%%%%%%%%%%%%%%%%%%%%%%%%%%%%%%%%%%%%%%%%%%%%%%%%%%%%%%%%

\maketitle

%%%%%%%%%%%%%%%%%%%%%%%%%%%%%%%%%%%%%%%%%%%%%%%%%%%%%%%%%%%%%%%%%%%%%%%%%%%%%%%

\setcounter{page}{1}
\section{Portfolio documentation}
\label{sec:documentation}

Compile a comprehensive documentation of your project, including all the project phases. You will need to explain every choice you made during the project and your thoughts about the results you get. You will introduce the results in suitable visualisation. Furthermore, you will need to explain which criteria you follow to build your prompts and how they affect the results. 

Students write the entire documentation with sections, sub-sections, diagrams, etc in this section. Please write as comprehensively as possible. Head to the document 1\_documentation.tex. You are free to use as many subsections as required. We will not provide a template for documentation. 

\section{Reflection}
\label{sec:reflection}

\textbf{In 3-5 pages, 1500-2000 words}\\

This section needs to be adjusted to align with the reflection requirements specified in the selected task.\\

\textbf{Note:} You should address all the questions from your selected task. Please list each question and provide your answers in the following enumeration.\\

For example:
\begin{enumerate}
    \item What was the most interesting thing you learned while working on the portfolio? What aspects did you find interesting or surprising? 
    
     \textbf{Answer:}
    
     \item

    
\end{enumerate}




 

 





\newpage
%  !!!!! Remove the following (until before \medskip command) before submitting !!!! :)
All of the resources used by the student to complete the portfolio task should be organised in the references section. 
\textbf {Note that the Reference section does not count towards the number of pages of the report.} Example references are given below \cite{einstein}\cite{knuthwebsite}\cite{latexcompanion}. \textbf{ If you are using a reference manager like Zotero, you can export your Zotero library as a .bib file and use it on Overleaf. As you cite the article/technology/library in your main text, the References section will automatically update accordingly.} Please include a full list of references found. If students are using Zotero for their research paper management, a bibTeX will help them during citation which automatically adds references to the report.  

\medskip

\bibliographystyle{unsrt}%Used BibTeX style is unsrt
\bibliography{references}

%%%%%%%%%%%%%%%%%%%%%%%%%%%%%%%%%%%%%%%%%%%%%%%%%%%%%%%%%%%%%%%%%%%%%%%%%%%%%%%
\end{document}
%%%%%%%%%%%%%%%%%%%%%%%%%%%%%%%%%%%%%%%%%%%%%%%%%%%%%%%%%%%%%%%%%%%%%%%%%%%%%%%